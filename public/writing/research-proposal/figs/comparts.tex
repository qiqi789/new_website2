\documentclass[12pt,border=11pt]{standalone}

\usepackage[UTF8]{ctex}

\usepackage{tikz}
\usetikzlibrary{arrows.meta, shapes,  shapes.geometric, calc, fit,positioning, backgrounds }
\tikzset{%
  >={Latex[width=2mm,length=2mm]},
  % Specifications for style of nodes:
            base/.style = {rectangle, rounded corners, draw=black,
                           minimum width=2cm, minimum height=1cm,
                           text centered, text width=7cm, font=\sffamily},
  activityStarts/.style = {base, fill=blue!30},
       startstop/.style = {base, fill=red!30},
    activityRuns/.style = {base, fill=green!30},
         process/.style = {base, minimum width=2.5cm, fill=orange!15,font=\ttfamily},
    basic box/.style = {
    shape = rectangle,
    align = center,
    draw  = #1,
    fill  = #1!18,
    rounded corners,
    inner sep=0.3cm},
}

\usepackage{enumitem}
%\setlist{noitemsep} 
\setlist{nosep}

\begin{document}

%\begin{tikzpicture}[node distance=1.5cm, every node/.style={fill=white}, align=center]
\begin{tikzpicture}[ align=center]

\node (instances) [base,align=left] at(2,18) {\kaishu 模拟实验实例的生成
 \begin{itemize}
\item $n = 3,6,12,...$
\item $m/n = 1,2,3,...$
\item 随机生成 $w_{i}\in(0,1),i\in\{1,...,n\},\sum w_{i}=1$
\item 随机生成价值矩阵$V=\{v_{ij}\}$, 二元和非二元值两种情况
\end{itemize}
};

\begin{scope}[on background layer]
    \node[fit = (instances), basic box = orange] (ship1) {};
\end{scope}

\node (algorithms) [base, align=left,below=of instances, ] {\kaishu 近似算法的实现%
\begin{itemize}
\item 基于组合结构和局部搜索的方法
\item 数学规划方法,包括凸优化技术
\end{itemize}
};

\begin{scope}[on background layer]
    \node[fit = (algorithms), basic box = green] (ship2) {};
\end{scope}

\node (metrics) [base, align=left, base right= of instances,] {\kaishu 评估指标%
\begin{itemize}
\item \textrm{NSW} 目标值 
\item \textrm{Pareto} 最优性
\item \textrm{EF1}, \textrm{EFX}, $(1-\epsilon)\textrm{-EFX}$ 公平属性 
\end{itemize}
};

\begin{scope}[on background layer]
    \node[fit = (metrics), basic box = blue] (ship3) {};
\end{scope}

\node (robustness) [base, align=left,below=of metrics] {\kaishu 鲁棒性评估%
\begin{itemize}
\item 高斯分布噪音,$\sigma \in (0,1)$,加入 $v_{ij}$ 或 $w_i$
\item 采样估计商品分配结果的变化概率
\end{itemize}
};

\begin{scope}[on background layer]
    \node[fit = (robustness), basic box = red] (ship4) {};
\end{scope}


\begin{scope}[on background layer]
    \node[fit = (robustness)(instances)(algorithms)(metrics)] (ship5) {};
\end{scope}


%\node (data)  [base] at(-1,7.5) {\kaishu 航空运营数据} ;
%\node (delayModel) [base, right of=data] at(4,7.5) {\kaishu 特定领域知识(概率图形模型,延误模型)} ;
%
%\node (propagation) [base, right of=delayModel, text width=2.6cm] at(8,7) {\kaishu 延误传递模型} ;
%\node (recovery) [base, right of=propagation] at(12,7) {\kaishu 恢复机制} ;
%
%\node (flights)  [base] at(-1,5) {\kaishu 航班规划数据} ;
%\node (model)  [base, text width=2.7cm] at(3,5) {\kaishu 鲁棒优化决策模型} ;
%\node (solution)  [base] at(7,5) {\kaishu 任务编排解} ;
%\node (metric)  [base] at(13,5) {\kaishu 鲁棒衡量指标} ;
%
% \begin{scope}[on background layer]
%    \node[fit = (flights)(model)(solution)(metric), basic box = blue] (mainprocess) {};
% \end{scope}
%
%\draw [->] (data) -- node[midway, above, text width=2cm]{\kaishu 数据挖掘、统计分析} (delayModel);
%\draw [->] (flights) -- (model);
%\draw [->] (delayModel) -- (model);
%\draw [->] (model) -- (solution);
%
%\draw [->] (solution) -- node(simulate)[midway, above] {\kaishu 模拟实验} (metric);
%\draw [->] (metric) --  (13,3)  -| node [near start, above] {\kaishu 评估鲁棒优化方法} (model);
%
%\draw [->] (propagation) -- (simulate);
%\draw [-, dashed] (recovery) -- (propagation);


\end{tikzpicture}

\end{document}